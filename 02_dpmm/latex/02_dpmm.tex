% To compile: pdflatex file.tex
\documentclass{article}
\usepackage{fullpage}
\usepackage{pgffor}
\usepackage{amssymb}
\usepackage{bm}
\usepackage{mathtools}
\usepackage{verbatim}
\usepackage{appendix}
\usepackage{graphicx}
\usepackage{url} % for underscore in footnote
\usepackage[UKenglish]{isodate} % for: \today
\cleanlookdateon                % for: \today
%\pagestyle{empty} % Removes page number. Graphs too big.

\def\wl{\par \vspace{\baselineskip}\noindent}
\def\beginmyfig{\begin{figure}[htbp]\begin{center}}
\def\endmyfig{\end{center}\end{figure}}
\def\prodl{\prod\limits_{j=1}^{n^*}}
\def\suml{\sum\limits_{i=1}^n}
\def\ds{\displaystyle}
\def\tu{\textunderscore}

\begin{document}
% my title:
\begin{center}
  %\section*{\textbf{Stat637 Homework 8}
  %  \footnote{https://github.com/luiarthur/Fall2014/blob/master/Stat637/8}
  %}  
  \section*{\textbf{UCSC AMS 241 Homework 1 - DP Priors}
    \footnote{\url{https://github.com/luiarthur/bnp_hw/01_dp_priors}}
  } 
  \subsection*{\textbf{Arthur Lui}}
  \subsection*{\noindent\today}
\end{center}

\noindent
\section{Correlation of $G(B_1)$ and $G(B_2)$}
Assume a Dirichlet process (DP) prior, DP($\alpha,G_0$), for distributions $G$
on $\mathcal X$ . Show that for any (measurable) disjoint subsets $B_1$ and
$B_2$ of $\mathcal X$ , Corr($G(B_1),G(B_2)$) is negative. Is the negative
correlation for random probabilities induced by the DP prior a restriction?
Discuss.\\

\noindent $Proof:$\\

\noindent
Let $B_3$ = $(B_1 \cap B_2)^C$, and $U_i = G(B_i)$, for $i=1,2,3$.
Furthermore, let $v_i = G_0(B_i)$, for $i=1,2,3$. Then, Corr($G(B_1),G(B_2)$) =
Corr($U_1,U_2$). Note that Sign(Corr($U_1,U_2$)) = Sign(Cov($U_1,U_2$)). So,
all we need to do is show that the covariance is negative. Finally, let
$f(u_1,u_2,u_3)$ be the pdf for the Dirichlet distribution, with parameters
$(\alpha v_1, \alpha v_2, \alpha v_3)$. \\
\[
  \begin{array}{rcl}
                            \vspace{.5em}
    \text{Cov}(U_1,U_2) &=& \text{E}[U_1 U_2] - \text{E}[U_1]\text{E}[U_2] \\
                            \vspace{.5em}
                        &=& \ds\int\int\int u_1u_2~f(u_1,u_2,u_3)~du_3~du_2~du_1 - v_1 v_2 \\
                            \vspace{.7em}
                        &=& \ds\int u_1u_2~
                            \frac{\Gamma(\alpha v_1 + \alpha v_2 + \alpha v_3)}
                            {\Gamma(\alpha v_1)\Gamma(\alpha v_2)\Gamma(\alpha v_3)}
                            u_1^{\alpha v_1 - 1} u_2^{\alpha v_2 - 1} u_3^{\alpha v_3 - 1}
                            ~d\mathbf u - v_1 v_2 \\
                            \vspace{.5em}
                        &=& \ds\frac{\Gamma(\alpha v_1 +1)\Gamma(\alpha v_2 +1)}
                            {\Gamma(\alpha v_1 + \alpha v_2 + \alpha v_3 + 2)}
                            \ds\frac {\Gamma(\alpha v_1 + \alpha v_2 + \alpha v_3)}
                            {\Gamma(\alpha v_1)\Gamma(\alpha v_2)} \times\\
                            \vspace{.9em}
                        &&  \ds\int\frac{\Gamma(\alpha v_1 + \alpha v_2 + \alpha v_3 + 2)}
                            {\Gamma(\alpha v_1+1)\Gamma(\alpha v_2+1)\Gamma(\alpha v_3)}
                            u_1^{(\alpha v_1 + 1) - 1} u_2^{(\alpha v_2 + 1) - 1} u_3^{\alpha v_3 - 1}
                            ~d\mathbf u - v_1 v_2 \\
                            \vspace{.7em}
                        &=& \ds\frac{\Gamma(\alpha v_1 +1)\Gamma(\alpha v_2 +1)}
                            {\Gamma(\alpha v_1 + \alpha v_2 + \alpha v_3 + 2)}
                            \ds\frac {\Gamma(\alpha v_1 + \alpha v_2 + \alpha v_3)}
                            {\Gamma(\alpha v_1)\Gamma(\alpha v_2)} - v_1 v_2\\
                            \vspace{.7em}
                        &=& \ds\frac{\Gamma(\alpha)}{\Gamma(\alpha+2)}
                            \ds\frac{\alpha v_1\Gamma(\alpha v_1) \cdot \alpha v_2\Gamma(\alpha v_2)}
                            {\Gamma(\alpha v_1)\Gamma(\alpha v_2)} - v_1 v_2\\
                            \vspace{.7em}
                        &=& \ds\frac{\alpha v_1 v_2}{(\alpha +1)} - v_1 v_2 \\
                            \vspace{.5em}
                        &=& v_1 v_2 \ds\frac{\alpha - \alpha - 1}{(\alpha +1)} \\
                            \vspace{.5em}
                        &=& -\ds\frac{v_1 v_2 }{(\alpha +1)} \\
                            \vspace{.5em}
                        &=& -\ds\frac{G_0(B_1) G_0(B_2) }{(\alpha +1)} \\
                        &<& 0 \\
  \end{array}
\]
Therefore, Corr($G(B_1),G(B_2)$) is negative. This result is intuitive and not necessarily a restriction.
We want it to be the case that as more weight is given to a partition $B_i$, then less weight is
given to the partition $B_j$, if $B_i$ and $B_j$ are disjoint.

\newpage
\section{Simulation of Dirichlet Process Prior Relizations}
Figure 1 shows draws from the DP using Setharuman's with varying $\alpha$
values. The $\alpha$ values from left to right are 1, 3, and 5. The mass
parameter $\alpha$ affects the point-wise variance of the cdf's drawn from the
DP. Notice that with small $\alpha$, the point-wise variance is larger at the
center of the DP draws; and with larger $\alpha$, the point-wise variance is
small. (This is illustrated by the red line.) The expected value of the DP is
the base-line distribution $G_0$, which is a standard Normal in our case.  To
confirm this, the point-wise means for the DP draws were taken and plotted in
blue. A few partiular draws from the DP are drawn in black, and the rest of the
draws are shaded in light grey such that the more likely draws are darker. \\

\noindent
Figure 2 shows draws from the DP using Ferguson's construction, again with
varying $\alpha$ values.  From left to right, the parameters are again 1, 3,
and 5. We see that there are no differences between the plots in Figure 1 and
2. This is expected as the 2 methods used to simulate draws from the DP should
yield the same distribution.

\vspace{5em}
\beginmyfig
  \includegraphics[scale=.4]{../code/pdfs/sethDP.pdf}
  \caption{DP draws using Setharuman's construction with varying $\alpha$
  values, with $\alpha=1, 3, \text{and } 5$ from left to right.}
\endmyfig
\beginmyfig
  \includegraphics[scale=.4]{../code/pdfs/fergusonDP.pdf}
  \caption{DP draws using Ferguson's construction with varying $\alpha$
  values, with $\alpha=1, 3, \text{and } 5$ from left to right.}
\endmyfig

\noindent
Figure 3 shows draws from a mixture of Dirichlet process (MDP), which is
different from a Dirichlet process mixture model. Priors were placed on the
$\alpha$ parameter and draws from the DP were made by sampling from the priors
for $\alpha$ and then sampling from a DP($\alpha,G_0) | \alpha$, where $\alpha
\sim \Gamma(a,b)$ and $b$ is the scale parameter for the Gamma distribution.
Again, $G_0$ was chosen to be a standard Normal. The density for $\alpha$ was
plotted in orange. The $(a,b)$ parameters are chosen such that the means and
variances for $\alpha$ from the left plot to the right plot are (1,1), (5,1),
and (5,1000). When both the mean and variance for $\alpha$ are small (1,1), the
point-wise variance is large in the center of the DP draws. With the same
variance but a slightly larger mean (5,1), the point-wise variance is smaller.
This is expected as on average, $\alpha$ is larger. And we learned from Figures
1 and 2 that larger $\alpha$ yields smaller point-wise variance. It is
interesting that with a really large variance (1000) but a mean of only 5, the
point-wise variance is again large. This is because for the Gamma distribution,
more weight is given to $\alpha$ values closer to 0, but values far out in the
right tail can be drawn frequently. This leads to a combination of DP draws
that are both very discrete and only slightly discrete.
\beginmyfig
  \includegraphics[scale=.37]{../code/pdfs/priorMDP.pdf}
  \caption{Mixture Dirichlet Process Draws. For each of these plots, 
  $G|\alpha \sim \text{DP}(\alpha,G_0)$, and $\alpha \sim \text{Gamma}(a,b)$,
  where $a$ and $b$ are respectively the shape scale parameters of the 
  Gamma distribution.}
\endmyfig

\section{Posterior Inference for One-sample Problems using DP Priors}
Data $\mathbf y = y_1,...,y_n$ for this section were generated from 
the two following distributions
\begin{enumerate}
  \item N(0,1) 
  \item $.5$N$(-2.5, .5^2 ) + .3$N$(.5, .7^2 ) + .2$N$(1.5, 2^2 )$\\
\end{enumerate}
with three different sample sizes, 20, 200, and 2000. The data were fit 
to the following model
\[
  \begin{array}{rclcl}
    y_i &|& G &\overset{i.i.d.}{\sim}& G,~~ i = 1, ..., n \\
        &&  G &\sim& \text{DP}(\alpha, \text{N}(m,s^2) )
  \end{array}
\]
where $\alpha$, $m$, and $s^2$ were each fixed at one of 2 values -- a small
value, and a large one. That is, $\alpha$ was either 1 or 10, $m$ was either 0
or 3, and $s^2$ was either 1 or 9. There are a $2^3=8$ combinations of
parameters, three sample sizes, and two models. So, there are a total of 48
combinations of parameters, sample sizers and models. I have plotted (for only
some of these combination) the empirical cdf of the data, centering
distribution $N(m,s^2)$, expected value of the DP draws $E[G|\mathbf y]$, the
95\% credible interval for $G|\mathbf y$, and the true distribution that
generated the data in the colors black, red, blue, grey, and green,
respectively. I will discuss a few observations. \\

% Difficult to write these two paragraphs... Work on it!!!
\noindent
First of all, the credible intervals get smaller as sample size increases. This
is expected as is usually the case. Figure 4 shows the results of the
simulation done under the N(0,1) distribution. Most of the graphs look the same
and interesting behavior can be observed when $n=20$ (small). In all cases, for
$n=20$, the tails of $G|\mathbf y$ is pulled towards the tails of their
respective baseline distributions. 

\noindent
For the first set of generated data, when $G_0$ and the data have the same
distribution, (left top corner), the posterior mean of the DP is very close to
the data, with small variance for both large and small $\alpha$ (1). This is also
the case when $\alpha$ is large (10). When the baseline distribution is mispecified,
uncertainty becomes great at the tail of the posterior distribution for small $\alpha$;
with large $\alpha$ and a mispecified baseline distribution, however, the posterior
distribution is pulled towards the prior.\\

\noindent
For the bimodal mixture of normals, the uncertainty is greater than that in the
previous problem in general, but greatest in the middle of the distribution
(where there is a trough in the density). This is because the baseline distribution
does not aid as much in identifying the true distribution, since the baseline distribution
is only unimodal. Interestingly, with a sample size of 20, and a baseline distribution 
of N(0,1), the credible intervals for the posterior distribution is smaller with larger
$\alpha$ than it is with smaller $\alpha$ (first and third panel in first row). The reason 
is not apparent to me, but I suppose it is because the baseline distribution and 
and distribution of the data share some common trends. Finally, the uncertainty for the
posterior when $\alpha$ is large and the baseline distribution is further from the 
true distribution is larger. And the posterior is pulled towards the baseline 
distribution because with a large $\alpha$, the posterior will borrow more from 
baseline distribution. \\

\beginmyfig
  \includegraphics[scale=.5]{../code/pdfs/dppost1.pdf}
  \caption{Posterior inference for one-sample problems using 
  DP priors. Note that the first, second, and third rows of th plots are for
  sample sizes 20, 200, and 2000, respectively.  Also, the black, red, blue, and
  green lines represent the empirical cdf, $G_0(m,s)$, E$[G|y]$ and the true
  underlying cdf of the generated data.}
\endmyfig
\beginmyfig
  %\includegraphics[scale=.5,angle=270]{../code/pdfs/dppost2.pdf}
  \includegraphics[scale=.5]{../code/pdfs/dppost2.pdf}
  \caption{Posterior inference for one-sample problems using 
  DP priors. Note that the first, second, and third rows of th plots are for
  sample sizes 20, 200, and 2000, respectively.  Also, the black, red, blue, and
  green lines represent the empirical cdf, $G_0(m,s)$, E$[G|y]$ and the true
  underlying cdf of the generated data.}
\endmyfig


\section{Posterior Inference for Count Data using MDP Priors}

\noindent
Data (with sample sizes of 300) were generated from two distributions,
\begin{enumerate}
  \item Poisson(5)
  \item .7 Poisson(3) + .3 Poisson(11).
\end{enumerate}
A simulation study was conducted using the above two simulated data sets to
explore the behavior of the mixture of DP's model. The MDP model used was
\[
  \begin{array}{rclcl}
    y_i &|& F &\overset{i.i.d}{\sim} & F, \text{ for } i=1,...,300 \\
    F &|& \alpha, \lambda &\sim& \text{DP}(\alpha,\text{Poisson}(\lambda)) \\
      && \alpha &\sim& \Gamma(2.5,2) \\
      && \lambda &\sim& \Gamma(5,1), \\
  \end{array}
\]
where the parameters used in the Gamma distribution are the shape and scale. 
The parameters (2.5,2) in the gamma prior for $\alpha$ correspond to a mean and
variance of 5 and 10. This is done so that weight is given to higher values of
$\alpha$ while ensuring $\alpha$ stays low. This is especially important for 
data 2 because we know from observing the data that the data is not unimodal.
And using a Poisson (unimodal) centering distribution in the DP with low
$\alpha$ will allow us to sample more from the empirical distribution of the 
data and less from the centering distribution. The parameters (5,1)
were used in the gamma prior for $\lambda$, which corresponds to a mean 
and variance of 5 and 5. The range of my data is between 0 and 31. So, a reasonable
estimate of the mean would be below 10 but the variance a 5 allows for deviance in
the sampling algorithm away from lower values for $\lambda$.\\

\noindent
We can explore the posterior distributions of $F, \alpha, \text{ and } \lambda$
by constructing a Gibbs sampler. At each step of our sampler, we update
$F$ by noting that the DP is a conjugate prior, and updating $F_{new}$ to be
DP($\alpha_{current} + 300, \widetilde{F_0}(\lambda_{current}) $), where 
\[
\widetilde{F_0}(\cdot) = 
\ds\frac{\alpha}{\alpha+n}F_0(\cdot|\lambda) + \frac{1}{\alpha+n} \suml\delta_{y_i}(\cdot).
\]
Next, to update the hyper parameters $\lambda$ and $\alpha$, we can use the
result proved by Antoniak to compute the marginal likelihood for DP priors with
discrete baseline distributions to be
\[
  L(\alpha, \lambda ; data) \propto 
  \ds\frac{\alpha^{n^*}}{\alpha^{(n)}} \prodl f_0(y_j^*|\lambda)
  \{\alpha f_0(y_j^*|\lambda) + 1\} ^{(n_j-1)}
\]
where 
\begin{itemize}
  \item $f_0(\cdot |\lambda)$ is the p.m.f. of $F_0(\cdot |\lambda)$
  \item $n^*$ is the number of distinct values in $(y_1,...,y_n)$
  \item $\{y_j^*: j = 1, ..., n^*\}$ are the distinct values in $(y_1,...,y_n)$
  \item $n_j = |\{i: y_i = y_j^*  \}|$, for $j = 1,..., n^*$
  \item notationally, $z^{(m)} = z(z + 1) \times ... \times (z + m - 1) = 
    \ds\frac{\Gamma(z+m)}{\Gamma(z)}$, for $m > 0$, with $z(0) = 1$.
\end{itemize}
Using this marginalized likelihood, we can use a Metropolis step to update our
parameters by computing an acceptance ratio based on the ratio of the product
of the likelihood and prior evaluated at the proposed candidate and the
previous value for our hyper parameters. \\

\noindent
Often, Bayesians will use a posterior predictive distribution to check model
validitiy. To obtain a posterior predictive distribution $F(y^*)|data$, we
first take our MCMC samples from $p(\alpha,\lambda) \propto
\pi(\alpha)\pi(\lambda) L(\alpha,\lambda; data)$, and simulate draws from
$F|\alpha,\lambda,\mathbf y$.  Since we are creating the posterior predictive
for discrete data, it is sensible to convert the obtained distribution into 
a p.m.f. \\

\beginmyfig
  \includegraphics[scale=.6]{../code/pdfs/postMDP1.pdf}
  \caption{Posterior inference for count data using MDP prior. Data generated from
  Poisson(5).}
\endmyfig

\beginmyfig
  \includegraphics[scale=.6]{../code/pdfs/postMDP2.pdf}
  \caption{Posterior inference for count data using MDP prior. Data generated from
  .7 Poisson(3) + .3 Poisson(11).}
\endmyfig

\noindent
The results are shown below in Figures 6 and 7. The top left panel shows the
data in green, the expected value of $G|\mathbf y$, in blue the posterior
predictive in red, and the 95\% credible intervals for $G|y$ in grey. The
posteriors appear to mimic the data, which is expected as we placed a small
mean for $\alpha$ in our prior for $\alpha$. The posterior distributions for
$\alpha$ and $\lambda$ are plotted in the next two panels along with their
trace plots.  The trace plots indicate that convergence is likely to have been
reached. The 95\% HPD's are shaded in blue and the means are drawn with a red
line. For the first data set, $\alpha$ was 14.6 which is relatively small when
considering the ratio $\frac{\alpha}{\alpha+n}= \frac{14.6}{314.6} \approx
.046$. But the variance of $\alpha$ is large, about 25. The mean for $\lambda$
is 5.29, with a variance of .2. This suggests that the MDP could approximate
$\lambda$ well when the baseline distribution is of the same form as the data.
In our case, the data were Poisson(5) and the baseline was a
Poisson($\lambda$). The posterior mean for $\lambda$ being 5 is reasonable.\\
Also, it appears from the joint distribution of $\lambda$ and $\alpha$ that the
two parameters are indeed not strongly correlated.\\

\noindent
For data 2, the mean and variance for $\alpha$ is smaller than that in data 1.
This is not surprising as a smaller $\alpha$ (on average) will lead to more
discreteness in the DP draws and less weight given to the baseline
distribution. This is desired because we know out data are actually bimodal and
not unimodal. The mean for $\lambda$ is larger (9.37). This is also expected
as our data were generated from a mixture of Poissons, which mean is 5.4. Once
again, it appears that $\alpha$ and $\lambda$ are not strongly correlated.

\end{document}
