% To compile: pdflatex file.tex
\documentclass{article}
\usepackage{fullpage}
\usepackage{pgffor}
\usepackage{amssymb}
\usepackage{bm}
\usepackage{mathtools}
\usepackage{verbatim}
\usepackage{appendix}
\usepackage{graphicx}
\usepackage[UKenglish]{isodate} % for: \today
\cleanlookdateon                % for: \today

\def\wl{\par \vspace{\baselineskip}\noindent}
\def\beginmyfig{\begin{figure}[htbp]\begin{center}}
\def\endmyfig{\end{center}\end{figure}}
%\def\prodl{\prod\limits_{i=1}^n}
\def\suml{\sum\limits_{i=1}^n}
\def\ds{\displaystyle}
\def\tu{\textunderscore}

\begin{document}
% my title:
\begin{center}
  %\section*{\textbf{Stat637 Homework 8}
  %  \footnote{https://github.com/luiarthur/Fall2014/blob/master/Stat637/8}
  %}  
  \section*{\textbf{AMS 241 Homework 1 - DP Priors}
  %  \footnote{https://github.com/luiarthur/bnp\_hw\01\_dp\_priors}
  } 
  \subsection*{\textbf{Arthur Lui}}
  \subsection*{\noindent\today}
\end{center}


\section*{Q1}
\subsection*{a:}
$\hat\phi = 1.294 > \frac{\chi^2_{n-p}(.95)}{n-p} =
\frac{\chi^2_{73}(.95)}{73} = 1.287$. At the 95\% confidence level, we
conclude that the there is overdispersion. The variance in the model is
higher than that if we do not model overdispersion.

\subsection*{b:}
The coefficient for group is significant. The log odds of death in the
treatment group is $1-e^{-.9289}=60.5\%$ \textbf{less} than that of the
control group. This is desired as the treatment is supposed to reduce the
likelihood of death.

\section*{Q2:}
\subsection*{a:}
The overdispersion parameter is $\hat\phi=0.05602$, which is significant at
the .05 $\alpha$ level. We conclude that there is overdispersion. The
variance of the parameters in the model is higher that that if we do not
model overdispersion.

\subsection*{b:}
The expected mean number of deaths in the treatment group is $1-e^{-.8522}
= 57\%$ \textbf{less} than that of the control group. This is expected for
the same reason in Q1. The effect of group is significant.

\section*{Q3:}
\subsection*{a:}
The overdispersion parameter is $\hat\phi=2.313 > 1.28$, which is
significant at the .05 $\alpha$ level. We conclude that there is
overdispersion. The variance of the parameters in the model is higher that
that if we do not model overdispersion.

\subsection*{b:}
The expected mean number of deaths in the treatment group is $1-e^{-.8754}
= 58\%$ \textbf{less} than that of the control group. This is expected for
the same reason in Q1. The effect of group is significant.

\section*{Q4:}
\subsection*{a:}
The overdispersion parameter is $\hat\phi=1.090 < 1.28$, which is not
significant at the .05 $\alpha$ level. We conclude that there is not
overdispersion. The variance of the parameters in the model is not significantly higher than that if we do not model overdispersion.

\subsection*{b:}
The expected mean number of deaths in the treatment group is $1-e^{-1.036}
= 65\%$ \textbf{less} than that of the control group. This is expected for
the same reason in Q1. The effect of group is significant.

\section*{Q4:}
I prefer the quasi-binomial model because we can make inference about the odds of death given a lamb received the treatment. I think this may be more useful than knowing the number of deaths because we can convert odds to probabilities, and use the probabilities to estimates of the number of deaths.

%\section*{Plots}
%\beginmyfig
%  \caption{}
%  %\includegraphics{../pics/postGibbs.pdf}
%\endmyfig

\end{document}
