% To compile: pdflatex file.tex
\documentclass{article}
\usepackage{fullpage}
\usepackage{pgffor}
\usepackage{amssymb}
\usepackage{bm}
\usepackage{mathtools}
\usepackage{verbatim}
\usepackage{appendix}
\usepackage{graphicx}
\usepackage{url} % for underscore in footnote
\usepackage[UKenglish]{isodate} % for: \today
\cleanlookdateon                % for: \today

\def\wl{\par \vspace{\baselineskip}\noindent}
\def\beginmyfig{\begin{figure}[htbp]\begin{center}}
\def\endmyfig{\end{center}\end{figure}}
%\def\prodl{\prod\limits_{i=1}^n}
\def\suml{\sum\limits_{i=1}^n}
\def\ds{\displaystyle}
\def\tu{\textunderscore}

\begin{document}
% my title:
\begin{center}
  %\section*{\textbf{Stat637 Homework 8}
  %  \footnote{https://github.com/luiarthur/Fall2014/blob/master/Stat637/8}
  %}  
  \section*{\textbf{AMS 241 Homework 1 - DP Priors}
    \footnote{\url{https://github.com/luiarthur/bnp_hw/01_dp_priors}}
  } 
  \subsection*{\textbf{Arthur Lui}}
  \subsection*{\noindent\today}
\end{center}

\noindent
\section*{Q1}
Assume a Dirichlet process (DP) prior, DP($\alpha,G_0$), for distributions $G$
on $\mathcal X$ . Show that for any (measurable) disjoint subsets $B_1$ and
$B_2$ of $\mathcal X$ , Corr($G(B_1),G(B_2)$) is negative. Is the negative
correlation for random probabilities induced by the DP prior a restriction?
Discuss.\\

\noindent $Proof:$\\

\noindent
Let $B_3$ = $(B_1 \cap B_2)^C$, and $U_i = G(B_i)$, for $i=1,2,3$.
Furthermore, let $v_i = G_0(B_i)$, for $i=1,2,3$. Then, Corr($G(B_1),G(B_2)$) =
Corr($U_1,U_2$). Note that Sign(Corr($U_1,U_2$)) = Sign(Cov($U_1,U_2$)). So,
all we need to do is show that the covariance is negative. Finally, let
$f(u_1,u_2,u_3)$ be the pdf for the Dirichlet distribution, with parameters
$(\alpha v_1, \alpha v_2, \alpha v_3)$. \\
\[
  \begin{array}{rcl}
                            \vspace{.5em}
    \text{Cov}(U_1,U_2) &=& \text{E}[U_1 U_2] - \text{E}[U_1]\text{E}[U_2] \\
                            \vspace{.5em}
                        &=& \ds\int\int\int u_1u_2~f(u_1,u_2,u_3)~du_3~du_2~du_1 - v_1 v_2 \\
                            \vspace{.7em}
                        &=& \ds\int u_1u_2~
                            \frac{\Gamma(\alpha v_1 + \alpha v_2 + \alpha v_3)}
                            {\Gamma(\alpha v_1)\Gamma(\alpha v_2)\Gamma(\alpha v_3)}
                            u_1^{\alpha v_1 - 1} u_2^{\alpha v_2 - 1} u_3^{\alpha v_3 - 1}
                            ~d\mathbf u - v_1 v_2 \\
                            \vspace{.5em}
                        &=& \ds\frac{\Gamma(\alpha v_1 +1)\Gamma(\alpha v_2 +1)}
                            {\Gamma(\alpha v_1 + \alpha v_2 + \alpha v_3 + 2)}
                            \ds\frac {\Gamma(\alpha v_1 + \alpha v_2 + \alpha v_3)}
                            {\Gamma(\alpha v_1)\Gamma(\alpha v_2)} \times\\
                            \vspace{.9em}
                        &&  \ds\int\frac{\Gamma(\alpha v_1 + \alpha v_2 + \alpha v_3 + 2)}
                            {\Gamma(\alpha v_1+1)\Gamma(\alpha v_2+1)\Gamma(\alpha v_3)}
                            u_1^{(\alpha v_1 + 1) - 1} u_2^{(\alpha v_2 + 1) - 1} u_3^{\alpha v_3 - 1}
                            ~d\mathbf u - v_1 v_2 \\
                            \vspace{.7em}
                        &=& \ds\frac{\Gamma(\alpha v_1 +1)\Gamma(\alpha v_2 +1)}
                            {\Gamma(\alpha v_1 + \alpha v_2 + \alpha v_3 + 2)}
                            \ds\frac {\Gamma(\alpha v_1 + \alpha v_2 + \alpha v_3)}
                            {\Gamma(\alpha v_1)\Gamma(\alpha v_2)} - v_1 v_2\\
                            \vspace{.7em}
                        &=& \ds\frac{\Gamma(\alpha)}{\Gamma(\alpha+2)}
                            \ds\frac{\alpha v_1\Gamma(\alpha v_1) \cdot \alpha v_2\Gamma(\alpha v_2)}
                            {\Gamma(\alpha v_1)\Gamma(\alpha v_2)} - v_1 v_2\\
                            \vspace{.7em}
                        &=& \ds\frac{\alpha v_1 v_2}{(\alpha +1)} - v_1 v_2 \\
                            \vspace{.5em}
                        &=& v_1 v_2 \ds\frac{\alpha - \alpha - 1}{(\alpha +1)} \\
                            \vspace{.5em}
                        &=& -\ds\frac{v_1 v_2 }{(\alpha +1)} \\
                            \vspace{.5em}
                        &=& -\ds\frac{G_0(B_1) G_0(B_2) }{(\alpha +1)} \\
                        &<& 0 \\
  \end{array}
\]
Therefore, Corr($G(B_1),G(B_2)$) is negative. This result is intuitive and not necessarily a restriction.
We want it to be the case that as more weight is given to a partition $B_i$, then less weight is
given to the partition $B_j$, if $B_i$ and $B_j$ are disjoint.




%\section*{Plots}
%\beginmyfig
%  \caption{}
%  %\includegraphics{../pics/postGibbs.pdf}
%\endmyfig

\end{document}
